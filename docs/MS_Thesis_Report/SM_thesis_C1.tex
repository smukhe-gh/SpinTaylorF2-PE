\chapter{Introduction}

The detection of gravitational waves from merging binary black holes by Advanced LIGO~\cite{Event_0,
Event_2} firmly inaugarated the era of gravitational wave astronomy. Future detections by Advanced
LIGO and other upcoming  gravitational wave observatories~\cite{KAGRA, Virgo, LIGO_india} would
include signals from mergering binary neutron star (BNS) or neutron star-black hole (NS-BH)
binaries, and would allow us to probe the strong field dynamical regime of gravity even further.

Gravitational waves encode information about the astrophysical sources that generate them. One of
the key goals of gravitational  wave astronomy is to faithfully estimate the parameters (masses and
spins of the components of the binary, source location in the sky, equation of state (if the signal
is from a BNS/NS-BH systems)) encoded in the detected signal. However, parameter estimation is a
non-trivial  problem -- it scales both in theoretical complexity and comptuational cost as more
physical details such as spin- induced precession, tidal effects  are added to the waveform model.
At the same time, accurate waveform models are crucial if one wants to faithfully recover the source
parameters.

Spin-induced precession is a generic feature of compact binaries. However, including the effects of
precession in a waveform model for parameter esitmation, significantly increases the computational
cost associated with waveform generation, and therefore the time taken to complete the analysis also
increases. Due to this, parameter estimation studies usually neglect precession. However, studies
that neglect precession is expected to suffer from systematic bias in the estimated parameters.
Therefore, there is a need to incorporate the effects of precession in a waveform model in a
computationally efficient manner.

The SpinTaylorF2 waveform~\cite{Lundgren2014} is a computationally efficent and  analytically-
tractable waveform model for gravitational waves from generic precessing binaries, applicable to
systems where one component of the binary has zero or negligible spin. The SpinTaylorF2 waveform,
therefore, can be applicable to NS-BH binaries since the neutron star in a binary is expected to
have negligible spin~\cite{Lundgren2014}. Further, the waveform model is expressed as a sum of 5
sidebands, and these sidebands capture the effects of precession i.e. become dominant when the
system is precessing or non-precessing. This has serious implications, since one can then use
individual sidebands for estimating the parameters of the binary, which would lead to a massive
reduction in computational cost. 

In this project, we deterimine the regions of parameter space, where
one would be able to use a sideband for esitmating parameters instead of the full SpinTaylorF2 waveform. We compute 
the signal to noise ratio of the SpinTaylorF2 in various regions of parameter space, and investigate the contribution
to the total SNR by the sidebands, using overlap computations.  


