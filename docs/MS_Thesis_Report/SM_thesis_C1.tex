\chapter{Introduction}

The spectacular detection of gravitational waves from merging binary black
holes by Advanced LIGO~\cite{Event_0, Event_2} has firmly inaugurated the era
of gravitational wave astronomy. Future detections of merging binary black
holes (BBH), binary neutron stars (BNS), or neutron star-black hole (NS-BH)
binaries by Advanced LIGO and other upcoming  gravitational wave
observatories~\cite{KAGRA, Virgo, LIGO_india} would enable us to probe the
strong-field, dynamical regime of gravity in much greater detail, allowing for
more stringent tests of the theory general relativity ~\cite{Berti2015}, and
would help in understanding the nature of matter at supra-molecular densities,
for example, in the core of neutron stars~\cite{Agathos,Chatziioannou}. In
addition, joint observations in the eletromagnetic and gravitational wave band
is expected to open up a new domain of multi-messenger astronomy that could
provide crucual insights into the progenitors of highly energy events such as
short gamma-ray bursts~\cite{Arun2014} and Type 1a or core-collapse
supernovae~\cite{Falta2011,Ott2013}.

\section{Gravitational radiation from compact binaries}

Gravitational waves are solutions to the vacuum Einstein's field equations,
generated by sources with a time-varying quadrupole moment. These waves couple
very weakly with matter, which makes them astrophysically very interesting,
but also makes them extremely difficult to detect. They carry away energy,
linear and angular momentum from the system, and therefore, a compact binary
radiating gravitational waves, would eventually merge~\cite{Peters1963},
unlike in the Newtonian case, where one has stable elliptical orbits.

The gravitational wave emission from a compact binary merger, such as, from a
BBH, BNS or NSBH system has a characteristic form, and three distinct phases
---the inspiral, the merger and the ringdown phases~\cite{Event_0}. The
inspiral waveform can be characterized as a chirp signal, i.e., it increases
both in amplitude and time throughout the inspiral phase. This is followed by
the merger, where one expects peak GW luminosity, and finally the ringdown
stage, where the final remnant emits GW radiation and settles to into
stationary state, which, in astrophysically realistic scenarios, would be a
Kerr black hole.

Gravitational waves encode information about the astrophysical sources that
generate them. 

\section{Parameter estimation of precessing binaries}

One of the key goals of gravitational wave astronomy is to faithfully estimate
the astrophysical parameters---masses and spins of the components of the
binary, source location in the sky, and the equation of state if the signal if
the source comprises of neutron stars---encoded in the gravitational wave
signal. Parameter estimation, however, is a non-trivial  problem; it scales
both in theoretical complexity and computational cost as more physical details
such as spin-induced precession and tidal effects are added to the waveform
model. Concretely, for precessing systems, the problem is two fold: the cost
of generating precessing waveforms is much higher than their non-precessing
counterparts, and second, parameter estimation studies for precessing systems
need a higher dimensional, and highly dense template bank, which leads to a
major increase in the overall computational cost~\cite{Nat2017}.

At the same time, accurate waveform models are crucial if one wants to
faithfully recover the source parameters; un-modeled parameters in the
waveform model leads to systematic biases in the estimation~\cite{Bias_1,
Bias_2}.  It is now known that studies that neglect precession are expected to
suffer from systematic bias in the estimated parameters; see~\cite{Bias_1,
Bias_3}  for a discussion. Spin-induced precession~\cite{Apostolatos1994} is a
generic feature of compact binaries.  Therefore, there is a need to
incorporate the effects of precession in a waveform model in a computationally
efficient manner.

\section{Why SpinTaylorF2?}

The SpinTaylorF2 waveform~\cite{Lundgren2014} is a computationally efficient and
analytically-tractable, frequency waveform model for gravitational waves from
generic precessing binaries that is applicable to a single-spin binary. The
model therefore, can be applied NS-BH binaries since the neutron star in a
binary is expected to have negligible spin~\cite{NSBH_upperlims, Brown2012,
Kramer}. The waveform can be expressed as a sum of five terms (referred to as
sidebands) and each sideband corresponds to a modulation of a leading order
function. The relative amplitude of the sidebands depend on the extent of
precession of the system: for a non-precessing signal, only the one sideband
($m=2$) survives i.e. has a non-zero amplitude, but when the system is
precessing, all of the sidebands develop a non-zero amplitude, and therefore
start contributing to the amplitude of the total waveform. One can therefore ask
whether it's possible to use a single (or a combination of) sideband(s) for
parameter estimation, instead of the full SpinTaylorF2 waveform, which would
result in  a massive reduction of computational cost.

In this project, we determine if such an approach is indeed possible, and the
regions of  parameter space where the approach would be applicable. We do this
by computing the signal-to-noise ratio (SNR) of the full SpinTaylorF2
waveform, and then investigating the fractional SNR contribution of each of
the sidebands to the total SNR, by computing the overlap~\cite{Creighton} of
the sideband with the full waveform, in different regions of the parameter
space. Further, we also exploit this complimentary, and use numerical fits
dividing the two regions, to put bounds on the spin-alignment parameters of
the source.k