%=========================================================================
% SM 4/2017
% Master's thesis Report IISER Thiruvananthapuram
% Investigations of SpinTaylorF2
%=========================================================================

\documentclass[12pt, a4wide]{report}

%--------------------------------------
% list packages
%--------------------------------------

\usepackage{amsthm,amssymb,mathrsfs,setspace,pstcol}
\usepackage{amsmath}
\usepackage{play}
\usepackage{epsfig}
\usepackage[nottoc]{tocbibind}
\usepackage[T1]{fontenc}
\usepackage{graphicx}
\usepackage{hyperref}
\usepackage{geometry}
\usepackage[en-US]{datetime2}
\usepackage[font={footnotesize}]{caption}
\usepackage{subcaption}
\usepackage{xcolor}
\usepackage{afterpage}
% \usepackage{minted}
\usepackage[parfill]{parskip}

%--------------------------------------
% setup packages
%--------------------------------------

\geometry{margin=1in}
% \geometry{bindingoffset=1cm}
\geometry{textwidth=390pt}

\hypersetup{
    colorlinks=true,
    linkcolor=violet,
    filecolor=violet,      
    urlcolor=violet,
    citecolor=violet,
}

%--------------------------------------
% New commands
%--------------------------------------

\renewcommand{\chaptermark}[1]{\markboth{#1}{}}
\renewcommand{\sectionmark}[1]{\markright{\thesection\ #1}}
\renewcommand{\baselinestretch}{1.5}

\newcommand\blankpage{%
    \null
    \thispagestyle{empty}%
    \addtocounter{page}{-1}%
    \newpage}

\input xy
\xyoption{all}

\begin{document}

%\pagenumbering{arabic} \setcounter{page}{1}

% %=========================================================================
% % TITLE PAGE
% %=========================================================================
% \begin{titlepage}
% \enlargethispage{10mm}
% \begin{center}
% \vspace*{10mm}

% \textbf{\textsc{\LARGE Investigating SpinTaylorF2: A Single Spin Precessing
% Frequency Domain Waveform}}\\

% \vspace*{10mm}

% A thesis submitted in partial fulfillment \\ of the requirements for the
% degree of \\

% \vspace{5mm}

% \textbf{\textsc{\Large Master of Science}}\\
% in \\
% {\large \bf Physics} \\
% \vspace{7mm}
% {{by}} \\ \vspace{2mm}
% {\textbf{\large Soham Mukherjee}}\\
% {\textbf{IMS12109}}\\

% \vspace*{6.5mm}

% \begin{figure}[h]
%   \begin{center}
%   \includegraphics[scale=0.15, angle=-90]{images/IISER_Logo.pdf}
%   \end{center}
% \end{figure}
% \vspace*{3.5mm}
% {to} \\ 
% \vspace{2mm}
% {\bf\large School of Physics} \\
% {\bf\large \mbox{Indian Institute of Science Education and Research}}\\
% {\bf\large Thiruvananthapuram, 695016, India}\\ 
% \vspace{2mm}
% \today
% \end{center}
% \end{titlepage}
% \clearpage

% % =========================================================================
% % CERTIFICATE
% % =========================================================================
% \pagenumbering{roman}\setcounter{page}{2}
% \begin{center}
% {\textbf{DECLARATION}}
% \end{center}
% %\thispagestyle{empty}

% \noindent
% This is to certify that the work contained in this project report entitled
% \textbf{``Investigating SpinTaylorF2: A Single Spin Precessing Frequency
% Domain Waveform''}  submitted by \textbf{Soham Mukherjee} (\textbf{Roll No:
% IMS12109}) to the Indian Institute of Science Education and Research,
% Thiruvananthapuram, in partial fulfillment of the requirements for the degree
% of {\bf Master of Science in Physics}, has been carried out by
% him under my supervision and that it has not been submitted elsewhere for the
% award of any degree.

% \vspace{4cm}

% \noindent Thiruvananthapuram, 695016 \hfill Dr. Archana Pai\\
% \noindent April 2013 \hfill (Project Supervisor)

% \clearpage

% % =========================================================================
% % ACKNOWLEDGEMENTS
% % =========================================================================
% \begin{center}
% \textbf{ACKNOWLEDGEMENTS}\\
% \end{center}

% Over the past year several people have provided me with invaluable help and
% support. They have made my time at IISER Thiruvananthapuram a bearable and
% productive experience.

% First and foremost, I would like to thank Archana Pai for agreeing to be my
% supervisor, and for suggesting and guiding me through every step of this
% project. Her support and encouragement has been instrumental towards my growth
% as a researcher, and my decision to continue in the field of GW astronomy. I'm
% also thankful to my colleagues K. Haris and Kumar Atmjeet, for the many useful
% discussions and brain-storming sessions we have had over the past year. I have
% also been extremely fortunate to meet Nathan Johnson-McDaniel at ICTS. Over
% the past year Nathan has been much more than a collaborator; he has been a
% mentor and a constant source of inspiration.

% Finally, I would like to expression my sincere gratitude to my friends
% Gayathri, Vinaya, and especially Navya, without whom, this journey would
% surely have been lonelier than usual. Navya is probably the reason why I even
% decided to jump into astrophysics, and I'd forever be thankful to her for
% sharing with me her enthusiasm for the subject.

% \clearpage

% % =========================================================================
% % ABSTRACT
% % =========================================================================
% \begin{center}
% \textbf{ABSTRACT}\\
% \end{center}

% We study the SpinTaylorF2 waveform---a closed-form, frequency domain waveform
% model that is ideally suited for describing the gravitational radiation from
% the inspiral phase of a single spin compact binary system, such as a neutron
% star--black hole binary. The quadrupole $(2,2)$ mode of the waveform model can
% be expressed as a sum of five spin-harmonics, where each spin harmonic
% corresponds to a modulation of a leading order function. We investigate the
% feasibility of using the sidebands of SpinTaylorF2 for inferring astrophysical
% parameters from gravitational waves emitted by a precessing neutron 
% star--black hole binary by computing the relative contribution of the individual
% spin-harmonics to the full waveform, and further, use this information to put
% bounds on the spin-alignment parameters of the astrophysical source.

% \clearpage
% \tableofcontents
% \listoffigures
% \newpage
% \pagenumbering{arabic}
% \setcounter{page}{1}

%=========================================================================
% CHAPTERS
%=========================================================================

% \input chapters/SM_thesis_C1.tex
% \input chapters/SM_thesis_C2.tex
% \input chapters/SM_thesis_C3.tex
\input chapters/SM_thesis_C4.tex
% \input chapters/SM_thesis_C5.tex
% \input chapters/SM_thesis_C6.tex

%=========================================================================
% CHAPTERS
%=========================================================================


\bibliographystyle{utphys.bst}
{\normalsize
\bibliography{SM_thesis_bib}}
\end{document}

