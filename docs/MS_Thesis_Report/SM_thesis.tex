%=========================================================================
% SM 11/2016
% Mid Term Report IISER Thiruvananthapuram
% Overlap analysis of SpinTaylorF2
%=========================================================================

\documentclass[12pt, a4wide]{report}

\usepackage{amsthm,amssymb,mathrsfs,setspace,pstcol}
\usepackage{amsmath}
\usepackage{play}
\usepackage{epsfig}
\usepackage[nottoc]{tocbibind}
\usepackage[T1]{fontenc}
\usepackage{graphicx}
\usepackage{hyperref}
\usepackage{geometry}
\usepackage[en-US]{datetime2}
\usepackage{mwe}
% \usepackage{subfig}
\usepackage{subcaption} 

%setup packages
\geometry{margin=1in}
% \geometry{bindingoffset=1cm}
\geometry{textwidth=390pt}

\hypersetup{
    colorlinks=true,
    linkcolor=darkgray,
    filecolor=blue,      
    urlcolor=blue,
    citecolor=blue,
}

\renewcommand{\chaptermark}[1]{\markboth{#1}{}}
\renewcommand{\sectionmark}[1]{\markright{\thesection\ #1}}

\input xy
\xyoption{all}

\theoremstyle{plain}
\newtheorem{theorem}{Theorem}[section]
\newtheorem{lemma}[theorem]{Lemma}
\newtheorem{corollary}[theorem]{Corollary}
\newtheorem{proposition}[theorem]{Proposition}

\theoremstyle{definition}
\newtheorem{definition}[theorem]{Definition}
\newtheorem{example}[theorem]{Example}
\newtheorem{notation}[theorem]{Notation}

\theoremstyle{remark}
\newtheorem{remark}[theorem]{Remark}

\renewcommand{\baselinestretch}{1.5}
\begin{document}

%\pagenumbering{arabic} \setcounter{page}{1}

%=========================================================================
% TITLE PAGE
%=========================================================================

\begin{titlepage}
\enlargethispage{10mm}
\begin{center}
\vspace*{-10mm}

\textbf{\textsc{\Large Analysis of SpinTalyorF2: A Single Spin Precessing Frequency Domain Waveform }}\\

\vspace*{10mm}

A Mid-Term Project Report Submitted \\
in Partial Fulfilment of the Requirements  \\
for the Degree of  \\

\vspace{5mm}

\textbf{\textsc{\Large Master of Science}}\\
in \\
{\large \bf Physics} \\
\vspace{7mm}
{\textit{by}} \\ \vspace{2mm}
{\textbf{\large Soham Mukherjee}}\\
{\textbf{IMS12109}}\\

\vspace*{5.5mm}

\begin{figure}[h]
  \begin{center}
  \includegraphics[scale=0.15, angle=-90]{./images/IISER_Logo.pdf}
  \end{center}
\end{figure}
\vspace*{3.5mm}
{\em to} \\ 
\vspace{2mm}
{\bf\large School of Physics} \\
{\bf\large \mbox{Indian Institute of Science Education and Research}}\\
{\bf\large Thiruvananthapuram -- 695016, India}\\ 
\vspace{2mm}
\DTMlangsetup{showdayofmonth=false}
\today
\DTMlangsetup{showdayofmonth=true}
\end{center}
\end{titlepage}
\clearpage


%=========================================================================
% CERTIFICATE
%=========================================================================

% \pagenumbering{roman} \setcounter{page}{2}
% \begin{center}
% {\large{\bf{CERTIFICATE}}}
% \end{center}
% %\thispagestyle{empty}


% \noindent
% This is to certify that the work contained in this project report
% entitled \textbf{``[Title of the project report]''}  submitted
% by \textbf{[Your Name]} (\textbf{Roll No: [Your roll number]}) to Indian Institute of Science Education and Research Thiruvananthapuram
% towards partial requirement of {\bf Master of Science} in Mathematics   has been carried out
% by [him/her] under my supervision and that it has not been submitted elsewhere
% for the award of any degree.


% \vspace{4cm}

% \noindent Thiruvananthapuram - 695 016 \hfill (Dr. XYZ)
% \noindent April  2013 \hfill Project Supervisor

% \clearpage

%=========================================================================
% ABSTRACT
%=========================================================================

% \begin{center}
% \textbf{ABSTRACT}\\
% \end{center}

% We study the SpinTaylorF2 waveform, a closed-form, single spin, frequency 
% domain waveform. The the dominant (2,2) mode can be further decomposed 
% into sidebands. The SpinTaylorF2 waveform is most suited for neutron star 
% black hole binaries, since in such systems the neutron star is expected to 
% have negligible spin compared to that of the black hole. In thus study, we 
% compare the contribution to the total SNR from the sidebands. We find that 
% (2,2,2) and (2,2,0) are dominant in complimentary regions of the parameter 
% space -- (2,2,2) mode dominates when there is no precession and (2,2,0) is 
% seen to dominate when the system is highly precessing. This would allow us 
% to use a particular sideband of the SpinTaylorF2 waveform, instead of the
% full SpinTaylorF2 (which more significantly more computationally expensive) 
% for parameter estimation. 

% \clearpage

% % \tableofcontents
% % \clearpage
% % \listoffigures
% % \listoftables

% \newpage

\pagenumbering{arabic}
\setcounter{page}{1}

%=========================================================================
% CHAPTERS
%=========================================================================

\input SM_thesis_C1.tex
\input SM_thesis_C2.tex

%=========================================================================
% CHAPTERS
%=========================================================================

\bibliographystyle{unsrt}
\bibliography{SM_thesis_bib}
\end{document}

