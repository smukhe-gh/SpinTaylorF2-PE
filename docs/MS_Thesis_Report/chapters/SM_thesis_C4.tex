%=======================================================================
% SM 4/2017
% Master's thesis Report IISER Thiruvananthapuram
% Investigations of SpinTaylorF2
%=======================================================================

\chapter{Bounds on astrophysical parameters}

In the previous chapter, we observed that the total SNR of the SpinTaylorF2
waveform is primarily divided between the $m=0$ and $m=2$ sidebands for a
large fraction of the spin-precession parameter space. Further, these two
sidebands dominate in almost complimentary regions of $(\theta_J, \kappa)$
parameter space: the $m=0$ mode appears to dominate for mildly precessing
systems, whereas $m=0$ mode dominates over the $m=2$ mode for strongly
precessing, edge-on binaries.

The fact that two sidebands capture most of the SNR over a large, but
complimentary regions of the parameter space, naturally leads to 3 possible
scenarios---the first two, where either $m=0$ or $m=2$ mode dominates over the
other, and the third, where both of these modes have comparable contribution
to the total SNR. In the sections below, we identify three regions $(\theta_J,
\kappa)$ space associated with strongly precessing, moderately precessing, and
mildly (non)-precessing systems, depending on the relative contribution of the
two sidebands to the full waveform.

\section{Partitioning the spin-precession parameter space}

\begin{figure}[!tp]
\centering
\includegraphics[width=0.8\linewidth]{images/OVLP_cut.pdf}
\caption{\small{Comparison of the overlaps, ${\cal{O}}_2$ and ${\cal{O}}_0$
for ${\cal{O}}_2, ~{\cal{O}}_0 > 0.4$ in the $(\theta_J, \kappa)$ parameter
space, for $m_{1}=14M_{\odot}$ and $\chi_1=0.8$. The three regions: region~(a)
where ${\cal{O}}_0 - {\cal{O}}_2 > 0.3$, region~(b) where ${\cal{O}}_2 -
{\cal{O}}_0 > 0.3$, and regions~(c) where $|{\cal{O}}_2-{\cal{O}}_0|< 0.3$.}}
\label{FIG:OVLP_cut_regions}
\end{figure}

We identified three regions in the $(\theta_J, \kappa)$ parameter space:
region~(a) where ${\cal{O}}_0$ is dominant $({\cal{O}}_0 - {\cal{O}}_2 >
0.3)$, region~(b) where ${\cal{O}}_2$ is dominant $({\cal{O}}_2 - {\cal{O}}_0
> 0.3)$, and finally, region~(c) where ${\cal O}_2$ and ${\cal O}_0$ are
comparable $(|{\cal{O}}_2-{\cal{O}}_0|< 0.3)$ to each other. We also enforced
the condition that both ${\cal O}_2$ and ${\cal O}_0$ are greater than or
equal to $0.4$ in all regions, in order to ensure both the sidebands are above
above the detectable threshold. The results is shown in
Fig.~\ref{FIG:OVLP_cut_regions}, which shows the three regions in the
$(\theta_J,\kappa)$ space for an NSBH system with BH mass $m_{1}=14 M_\odot$,
$\chi_1=0.8$, $\psi_J=0.001$ and $\alpha_0 =0.001$.


We approximated the boundaries of region~(c) by using two parabolas symmetric
around $\theta_J=\pi/2$, as functions of $\eta$ and $\chi_1$:
\begin{equation}
\kappa(\eta, \chi_1) = \kappa_{0}(\eta, \chi_1) - C(\eta, \chi_1)(\theta_J-\pi/2)^2, 
\label{EQ:boundary_region}
\end{equation}
where $\kappa_{0}(\eta, \chi_1)$ and $C(\eta, \chi_1)$ is given by
\begin{align}
\begin{split}
\kappa_{0a}(\eta, \chi_{1})~&\approx  (145.83 \chi_{1} - 155.92)\, \eta^2 -
(1.10 \chi_{1} + 0.16)\, \eta+ 0.08 \, \chi_{1}+0.50,\\
\kappa_{0b}(\eta, \chi_{1})~&\approx (48.98\chi_{1} - 57.13) \,\eta^2 +
(1.96{\chi_{1}}-2.39)\, \eta - 0.09 \, \chi_{1} +0.82,
\label{EQ:kappa_boundary}
\end{split}
\end{align}
\vspace{-10mm}
\begin{align}
\begin{split}
C_{a}(\eta, \chi_{1})~&\approx  (-6.23{\chi_{1}} + 10.47)\,
\eta +  0.01\, \chi_{1}+ 0.72,\\
C_{b}(\eta, \chi_{1})~&\approx  (-8.37{\chi_{1}} + 10.62)\, \eta +  0.10\,
\chi_{1}+ 0.30,
\label{EQ:phi_boundary}
\end{split}
\end{align}
where the subscript `$a$' and `$b$' represent the boundaries of region~(a) and
region~(b) with region~(c), respectively. We obtained the expressions for
$\kappa_{0}(\eta, \chi_1)$ and $C(\eta, \chi_1)$ by fitting a quadratic
function to the points on the boundaries of these regions over a range of
values for $\eta$ and $\chi_1$ ($0.08 < \eta < 0.24$ and $0 < \chi_1 < 1$).
As the extent of precession increases (i.e., for higher values of BH
mass/spin), the inner boundary that separates region~(b) and region~(c) shifts
to higher values of $\kappa$, since the contribution of the $m=0$ mode
increases. These fits, therefore, can used to probe the extent of precession
in the system, depending on where the system lies in the $(\theta_J,
\kappa)$ space.

\section{Upper bound on $\kappa$}

Figure ~\ref{FIG:OVLP_cut_regions} immediately suggests that there exists an
upper bound on the spin-alignment parameter $\kappa$ for strongly and
moderately precessing systems, i.e., systems in in region~(a) and region~(c).
In the case where the $m=0$ sideband has a significantly larger SNR than
$m=2$, we can assert that the system lies in region~(a) in
Fig.~\ref{FIG:OVLP_cut_regions}, which implies that the system is strongly
precessing. We can then use
Eqs.~(\ref{EQ:boundary_region})--(\ref{EQ:phi_boundary}), given we are able to
estimate $\eta$ using the total and the chirp mass of the binary, to bound the
spin orientation $\kappa$, depending on the BH spin. As an example, in
Fig.~\ref{FIG:kappa_max_bounds} we show the allowed values for $\kappa$ when
$m_1 = 14 M_{\odot}$ or $\eta=0.08$.  Note that even for $\chi_1=1$ (highest
possible BH spin), there exists an upper bound on $\kappa$ for strongly
precessing systems $(\kappa\leq 0.42)$, as well as for systems that show
moderate precession $(\kappa
\leq 0.64)$.

\begin{figure}[!htbp]
\centering
\includegraphics[width=0.6\linewidth]{images/kappa_max_bound.pdf} 
\caption{\small{The dashed line corresponds to the maximum value of $\kappa$ for
    a strongly precessing system ($m_1 = 14 M_{\odot}, m_2 = 1.4 M_{\odot}$),
    i.e, region~(a) and moderately precessing systems that lie in region~(b)
    in Fig.~\ref{FIG:OVLP_cut_regions}.}}
\label{FIG:kappa_max_bounds}
\end{figure}
